\documentclass[
	12pt
] {article}

\usepackage[
	a4paper,
	left=1.5cm,
	right=1.5cm,
	top=2.5cm,
	bottom=2.5cm,
	headheight=1cm
] {geometry} %for setting margin

\usepackage{fancyhdr} %for header and footers
\usepackage{parskip} %for leaving some space after paragraphs


% Math
\usepackage{mathtools} %helps to number equations
\usepackage{siunitx} % Units of measurement


% For graphs and advanced diagrams
\usepackage{tikz} %the drawing library
\usetikzlibrary{positioning} %some auxilary to the drawing library
\usetikzlibrary{math} %helps to define variables in the tikz

\usepackage{authblk} %authors and their affiliations

% Basic graphics
\usepackage{float} %to force the figure to be in a place among the text
\usepackage{graphicx} %for include graphics
\usepackage{caption} %better referencing to labels
\usepackage{subcaption}  %helps with multiple pictures in a figure


% Language and Font package: COMMENT THESE FOR ENGLISH ONLY
\usepackage[T2A]{fontenc} %[RUS] loads cyrillic characters
\usepackage[russian]{babel} %[RUS] changes date heading 'abstract' to Russian
\usepackage[utf8]{inputenc} %[RUS] to force utf-8 encoding
\usepackage{hyperref}


%-----------------------------------------------------------------------

\renewcommand\thesubfigure{\asbuk{subfigure}} %[RUS] for renaming subfigures to russian
\sisetup{output-decimal-marker = {,}} %[RUS] changes all ordinary decimal separator to rassiky commas
\pagestyle{fancy}
\graphicspath{{figures/}}


\newcommand\titleshort{Flow in Porous Media}

\author[1]{Шаббир Кафи Ул}

\affil[1]{Московский Физико-Технический Институт (национальный исследовательский университет)}

\title{Сетевая модель двухфазной фильтрации в неоднородных пористых среда}
\fancyhead{}
\fancyhead[L]{Manual}
\fancyhead[C]{\titleshort}
\fancyhead[R]{\the\year}


\sloppy
\begin{document}

\maketitle
\renewcommand{\arraystretch}{1.4}

\begin{abstract}
	The network model for simulation in porous media.
\end{abstract}

\tableofcontents

\section{Code documentation}
\subsection{File and Folder Structure}

	\begin{enumerate}
		\item \textbf{porous-fluid/docs/} documents, instructions and other files related to the model.
			\begin{enumerate}
				\item \textbf{instructions/} cmd commands
				\item \textbf{manual/} the documentation of code
				\item \textbf{publications/} list of papers and conference papers published on this model
				\item \textbf{reference-papers/} papers read to develop this model
				\item \textbf{reports/} reports, analysis and other calculation files
			\end{enumerate}

		\item \textbf{porous-fluid/makefile-gen/} the make file generator program
			\begin{enumerate}
				\item \textbf{build/} makefile-gen.exe and other object files
				\item \textbf{src/} the cource code of the makefile-gen.exe
				\item \textbf{file-structure.txt} the structure and list of files in porous-fluid/src
				\item \textbf{Makefile} commands to generate Makefile and put it in the root folder.
			\end{enumerate}
		\item \textbf{porous-fluid/run} this normally does not exist but when the code is run for the first time it is created automatically
			\begin{enumerate}
				\item \textbf{build/} all the object files
				\item \textbf{input/} incongen.txt, parameter.txt, tlength.txt, tmns.txt, tradius.txt
				\item \textbf{output/}
					\begin{enumerate}
						\item \textbf{calculations/} saturation-vs-x.txt
						\item \textbf{data-raw/} tpressure.txt tmns.txt tvelocity.txt ttime.txt
						\item \textbf{graph/} plots of saturation-vs-x.txt at various moments of time
						\item \textbf{initial-conditions/} tradius.txt, tlength.txt
						\item \textbf{pressure/}
						\item \textbf{velocity/}
						\item \textbf{visualization-nothick/} visualization where the thickness of the tube is visible
						\item \textbf{visualization-thick/}
					\end{enumerate}
				\item \textbf{incongen.exe}
				\item \textbf{test.exe}
				\item \textbf{simulate.exe}
			\end{enumerate}
	\end{enumerate}

\subsection{Makefile}
	The make file has the following high level commands:
	\begin{enumerate}
		\item \textbf{all}: necessary\_compile, run
		\item \textbf{necessary\_compile}: folder\_check, list\_exe
		\item \textbf{folder\_check}: check goodness of porous-fluid/run/ folder
		\item \textbf{run}: run\_exe, back\_up
		\item \textbf{force}: clean, necessary\_compile
		\item \textbf{clean}: cleans the build folder
		\item \textbf{list\_exe}: instructions on how to link
		\item \textbf{list\_obj}: instructions on how to build the object files
	\end{enumerate}

	This is a blank citation \cite{wikipedia-gini}
\bibliographystyle{plain.bst}
\bibliography{reference}
\end{document}
